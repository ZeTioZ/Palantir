\fancyhead[L]{Guide Utilisateur}
\fancyhead[C]{Foire Aux Questions}

\section{Foire Aux Questions}
\label{sec:faq}

\subsection{Pouvons-nous créer des stratégies par map ?}
Malheureusement, il n'est pas encore possible de faire une stratégie en particulier pour chaque map. Cependant, c'est une fonctionnalité qui est en cours de développement.

\subsection{On me demande des ressources que je ne vois pas.}
Il se peut que, même après exploration de la carte avec un éclaireur, la stratégie de base ne soit pas adaptée à 100\% à la carte que vous jouez. Ainsi, si il vous demande par exemple de chercher des buissons et qu'il n'y en a pas aux alentours, comprenez qu'ils vous faut récupérer de la nourriture au plus vite. Certaines map sont aussi dépourvue de points d'eau ce qui est parfois le seul point fort d'une civilisation, dans ce cas, la stratégie de base ne sera pas suffisant pour cette map. Nous ajouterons des stratégies particulières à chaque map dans de prochaines mises à jours.

\subsection{Certaines stratégies ne me semblent plus d'actualité.}
Le jeu est en constante évolution, il est de ce fait possible que certaines stratégies de base ne soient plus d'actualité et qu'il vaudrait mieux les mettre à jour, pour cela vous pouvez \hyperref[sec:configfiles]{créer votre stratégie} ou nous \href{https://github.com/UNamurCSFaculty/2223_INFOB318_Palantir/issues}{ouvrir un ticket sur le repository GitHub} pour que l'on puisse la repasser en revue.

\subsection{Comment puis-je soumettre ma propre stratégie ?}
Si vous souhaitez mettre à jour ou soumettre une meilleure stratégie pour une civilisation, vous pouvez directement créer un P.R. (Pull Request) directement sur notre \href{https://github.com/UNamurCSFaculty/2223_INFOB318_Palantir}{repository GitHub !}

\subsection{Je ne comprend pas grand chose aux guides.}
Bien que ce soient des guides pour débutants, il est évident qu'il faut au moins suivre le tutoriel du jeu pour comprendre les mécaniques de base du jeu et pouvoir comprendre les informations affichées par l'overlay. Nous vous conseillons donc de faire toutes la série d'entrainement disponible dans le mode campagne.

\subsection{Est-ce que je peux me faire bannir si j'utilise Palantir ?}
Non. Palantir n'est pas un logiciel de triche. Nous avons tout fait pour que l'anti-cheat du jeu ne puisse pas détecté un faux positif avec Palantir. Palantir ne fournit que des informations basées sur le visuel que peut apporter le jeu. Il ne vous montre pas d'informations cachées dans la mémoire du jeu ou d'autres informations qui constituerait un avantage excessif par rapport aux autres joueurs.

\subsection{J'ai trouvé un bug, comment puis-je le signaler ?}
Pour nous signaler un bug, vous pouvez simplement \href{https://github.com/UNamurCSFaculty/2223_INFOB318_Palantir/issues}{ouvrir un ticket sur le repository GitHub} avec votre version de Palantir ainsi que les étapes pour reproduire le bug, nous pourrons ensuite le corriger.

\subsection{Suis-je obligé de mettre mon HUD à 125\% ?}
Oui. Palantir utilise un modèle de reconnaissance de texte qui nécessite une certaine grandeur du texte pour garantir une précision de détection du texte en jeu. Sans cela, les zones de détection du texte se retrouvent déplacée et la précision de détection du texte se trouve détériorée.

\subsection{Puis-je mettre mon jeu en français ?}
Non. Palantir n'est actuellement disponible que pour la version du jeu en Anglais. En effet, nous utilisons les noms anglais des civilisations et des âges pour la détection et l'affichage des stratégies.

\subsection{Informations de contact}
Vous pouvez nous contacter avec les informations suivantes:\\

Email: \href{mailto:support@donatog.tech}{support@donatog.tech}\\
GitHub: \href{https://github.com/UNamurCSFaculty/2223_INFOB318_Palantir}{Palantir}\\