\fancyhead[L]{Guide du développeur}
\fancyhead[C]{Qu'est-ce que Palantir ?}

\section{Qu'est-ce que Palantir ?}
Palantir est le companion de jeu pour les débutants de Age Of Empire II: Definitive Edition.\\

Avec Palantir, vous passerez plus de temps à découvrir les mécaniques avancées du jeu qu'à vous attarder sur les détails de débutants.

\subsection{Comment cela fonctionne ?}
Palantir combine de l'intelligence artificielle et une base de stratégies provenant de guides et forums de Age Of Empire II: Definitive Edition. Il met en relation toutes ces informations stockées pour pouvoir vous afficher, en temp réel et en superposition au jeu, non seulement le meilleur contre pour la sélection de team actuelle, mais aussi la stratégie de base à suivre pour ne pas être en retard durant la partie.

\subsection{En quoi cet overlay me rend un meilleur joueur ?}
Palantir va vous alléger la lourde tâche d'adaptation au jeu et à ses stratégies de bases ainsi qu'à ses timings serrés que les débutants n'acquièrent que plus tard dans le jeu. Il va donc vous permettre de vous attarder aux spécialités des civilisations que vous jouez, mais aussi vous donner plus de temps pour vous familiariser avec l'interface du jeu sans pour autant être totalement perdu par ce qui se passe à l'écran. En somme, vous prenez du plaisir à jouer dés le début, sans devoir attendre d'en devenir un expert.

\subsection{Quel est l'intérêt de ce guide du développeur ?}
Ce guide va vous permettre de comprendre comment fonctionne chaque partie du code de l'API de l'application. Notez bien que cette documentation n'est qu'une petite extension à la documentation déjà présente sur le code. Dans cette documentation, vous y trouverez quelques exemple d'utilisation pour chacune des fonctions de l'API. L'API de l'interface graphique (wxPython) n'est pas expliquée ici. Pour cela, referez-vous directement à la documentation de ce module python. De ce fait, énormément de choses (notamment sur les structures des fichiers) sont déjà décrites dans le guide de l'utilisateur.