\fancyhead[L]{Guide du développeur}
\fancyhead[C]{Parseur de texte sur image}

\section{Parseur de stats}
\subsection{Fonction "get\_stats"}
Cette fonction, vous permet de récupérer depuis une capture d'écran du jeu en 1920*1080 et le HUD en 125\% de récupérer les informations relatives aux statistiques du joueur. Dans ces statistiques nous avons, le bois, la nourriture, l'or, la pierre et la population. Ces données vous serons renvoyée sous forme d'une liste avec ce format: [Bois, Nourriture, Or, Pierre, Population Actuelle/Population Max].
Exemple d'utilisation:

\begin{minted}{python}
stats = get_stats()
print(f"Wood: {stats[0]}")
print(f"Food: {stats[1]}")
print(f"Gold: {stats[2]}")
print(f"Stone: {stats[3]}")
print(f"Population: {stats[4]}")
\end{minted}

\subsection{Fonction "get\_villagers\_count"}
Cette fonction vous renvoie sous format de chaîne de caractères, le nombre de villageois que vous avez en ce moment d'après la capture d'écran en 1920*1080 et le HUD en 125\%.

\begin{minted}{python}
villagers = get_villagers_count()
print(f"Villagers: {villagers}")
\end{minted}

\subsection{Fonction "get\_actual\_age"}
Cette fonction vous renvoie sous format de chaîne de caractères, l'âge de votre civilisation que vous avez en ce moment d'après la capture d'écran en 1920*1080 et le HUD en 125\%.
Exemple d'utilisation:

\begin{minted}{python}
civ_age = get_actual_age()
if civ_age == "Dark Age":
    print("You are in the Dark Age !")
\end{minted}