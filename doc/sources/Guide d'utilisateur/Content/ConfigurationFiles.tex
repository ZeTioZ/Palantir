\fancyhead[L]{Guide Utilisateur}
\fancyhead[C]{Fichiers de configuration}

\label{sec:configfiles}
\section{Fichiers de configuration}

Palantir offre un système de fichiers de configuration qui vous permettra de créer vos propre stratégie et de les suivre en jeu grâce à l'\hyperref[sec:overlays]{\textbf{\textit{overlay}}}.

\subsection{Structure des fichiers de configuration}
Chaque fichier de configuration est propre à une civilisation. À l'heure actuelle, il n'est possible d'avoir qu'un seul fichier de configuration pour chacune des civilisations. Dans de prochaines mise à jour, la possibilité de choisir un fichier parmi plusieurs pour une même civilisation sera ajoutée.
\\
La structure du fichier est simple à comprendre et se présente de la façon suivante:
\begin{minted}{TEXT}
{
    "description": "Brève description de la civilisation avec ses points forts/négatifs."
    "speciality": "Spécialité d'une civilisation."
    "dark_age": [
        "Page 1 de la stratégie pour l'âge sombre.",
        "Page 2 de la stratégie pour l'âge sombre."
    ]
    "feudal_age": [...]
    "castel_age": [...]
    "imperial_age": [...]
    "thresh_holds": {
        "villagers": {
            "dark_age": Recommandé pour l'âge sombre
            "feudal_age": Recommandé pour l'âge feudal
            "castle_age": Recommandé pour l'âge des chateaux
        }
        "food": {
            "dark_age": Recommandé pour passer à l'âge feudal
            "feudal_age": Recommandé pour passer à l'âge des chateaux
            "castle_age": Recommandé pour passer à l'âge impérial
        }
        "wood": {...}
        "gold": {...}
        "stone": {...}
    }
}
\end{minted}

Il est à noté que le nom du fichier doit être le nom \textbf{exact} (majuscule comprise) de la civilisation à lequel il appartient. Par exemple, un fichier de configuration pour la civilisation des Azetcs serait "Aztecs.json".